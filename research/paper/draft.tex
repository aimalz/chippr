%\documentclass[iop]{emulateapj}
\documentclass[preprint]{aastex}
%\documentclass[12pt, onecolumn]{emulateapj}
%\documentstyle[aas2pp4,natbib209]{article}

\usepackage{tikz}
\usepackage{natbib}
\usepackage{amsmath}

\usetikzlibrary{shapes.geometric, arrows}
\usetikzlibrary{fit}

\tikzstyle{hyper} = [circle, text centered, draw=black]
\tikzstyle{param} = [circle, text centered, draw=black]
\tikzstyle{data} = [circle, text centered, draw=black, line width=2pt]
\tikzstyle{arrow} = [thick,->,>=stealth]

\newcommand{\myemail}{aimalz@nyu.edu}
\newcommand{\textul}{\underline}

\shorttitle{How to estimate the redshift density function}
\shortauthors{Malz and Hogg}

\begin{document}

\title{How to estimate the redshift density distribution}

\author{Alex Malz\altaffilmark{1} \& David W. Hogg\altaffilmark{1,2,3,4}}
\email{aimalz@nyu.edu}

\altaffiltext{1}{Center for Cosmology and Particle Physics, Department of Physics,
  New York University, 4 Washington Pl., room 424, New York, NY 10003, USA}
\altaffiltext{2}{Simons Center for Data Analysis, 160 Fifth Avenue, 7th floor, New York, NY 10010, USA}
\altaffiltext{3}{Center for Data Science, New York University, 726 Broadway, 7th floor, New York, NY 10003, USA}
\altaffiltext{4}{Max-Planck-Institut f\"ur Astronomie, K\"onigstuhl 17, D-69117 Heidelberg, Germany}

\begin{abstract}
The redshift density function $n(z)$ is a crucial ingredient for weak lensing cosmology.  Spectroscopically confirmed redshifts for the dim and numerous galaxies observed by past, present, and future weak lensing surveys are inaccessible, making photometric redshifts (photo-$z$s) the next best alternative.  Due to their intrinsic scatter and susceptibility to catastrophic outliers, photo-$z$ point estimates are being superseded by photo-$z$ probability distribution functions (PDFs).  However, analytic methods for utilizing these new data products in cosmological inference are still evolving.

This paper presents a novel approach to estimating the posterior distribution over $n(z)$ from a survey of galaxy photo-$z$ PDFs based upon a probabilistic graphical model.  We present the Cosmological Hierarchical Inference with Probabilistic Photometric Redshifts (CHIPPR) code implementing this technique, as well as its validation on mock data and testing on a subset of BOSS DR10.  The code and method are easily extensible to other one-point statistics of redshift that are informative to galaxy evolution and large-scale structure.

%Ongoing and upcoming galaxy surveys aiming to constrain cosmological parameters will produce photometric redshift (photo-$z$) probability distribution functions (PDFs), which are, generally, posteriors under an interim prior.  It is desirable to use this highly informative data product to estimate redshift-dependent distribution functions such as the redshift density function, stellar mass function, or We present a mathematically rigorous approach to obtain the posterior distribution on a one-point statistic of redshift from a survey providing photo-$z$ PDFs and their interim prior by way of a probabilistic graphical model corresponding to a hierarchical Bayesian relation of photo-$z$ PDFs to a posterior distribution over the one-point statistic.  This fully consistent method is applied to the redshift density function $n(z)$ necessary for the calculation of cosmological parameters from weak lensing surveys, validated on synthetic data and tested on a small subset of BOSS DR10 data.  It is demonstrated that this method improves accuracy, both qualitatively and as quantified by the Kullback-Leibler Divergence, compared to popular alternatives (such as stacking of photo-$z$ PDFs and conversion of photo-$z$ PDFs to point estimators of redshift).  The method can be straightforwardly generalized to other one-point statistics of redshift including distribution functions over luminosity, mass, star formation rate, etc.  We publicly release the Cosmological Hierarchical Inference with Probabilistic Photometric Redshifts (CHIPPR) code that provides a general framework for estimating the posterior distribution of a general one-point statistic of redshift.

\end{abstract}

\keywords{methods: analytical "\_\_\_" methods: data analysis "\_\_\_" methods: statistical "\_\_\_" techniques: photometric "\_\_\_" galaxies: statistics}

\section{Introduction}



\label{sec:intro}

%\appendix{}

\begin{acknowledgements}
AIM thanks Mohammadjavad Vakili for insightful input on statistic, Geoffrey Ryan for assistance in debugging code, and Boris Leistedt for helpful comments provided in the preparation of this paper.
\end{acknowledgements}

%\bibliographystyle{apj}
%\bibliography{references}

\end{document}