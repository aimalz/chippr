\documentclass[preprint]{aastex}

\usepackage{tikz}
\usepackage{natbib}
\usepackage{amsmath}

\usetikzlibrary{shapes.geometric, arrows}
\usetikzlibrary{fit}

\tikzstyle{hyper} = [circle, text centered, draw=black]
\tikzstyle{param} = [circle, text centered, draw=black]
\tikzstyle{data} = [circle, text centered, draw=black, line width=2pt]
\tikzstyle{arrow} = [thick,->,>=stealth]

\newcommand{\myemail}{aimalz@nyu.edu}
\newcommand{\textul}{\underline}
\newcommand{\chippr}{\texttt{CHIPPR} }

\shorttitle{How to obtain the redshift distribution from probabilistic redshift 
estimates}
\shortauthors{Malz and Hogg}

\begin{document}

\title{How to obtain the redshift distribution from probabilistic redshift 
estimates}

\author{Alex Malz\altaffilmark{1} \& David W. Hogg\altaffilmark{1,2,3,4}}
\email{aimalz@nyu.edu}

\altaffiltext{1}{Center for Cosmology and Particle Physics, Department of 
Physics,
  New York University, 726 Broadway, 9th floor, New York, NY 10003, USA}
\altaffiltext{2}{Simons Center for Computational Astrophysics, 162 Fifth 
Avenue, 7th floor, New York, NY 10010, USA}
\altaffiltext{3}{Center for Data Science, New York University, 60 Fifth Avenue, 
7th floor, New York, NY 10003, USA}
\altaffiltext{4}{Max-Planck-Institut f\"ur Astronomie, K\"onigstuhl 17, D-69117 
Heidelberg, Germany}

\begin{abstract}
The redshift distribution $n(z)$ is a crucial ingredient for weak lensing 
cosmology.  Spectroscopically confirmed redshifts for the dim and numerous 
galaxies observed by weak lensing surveys are expected to be inaccessible, 
making photometric redshifts (photo-$z$s) the next best alternative.  Because 
of the nontrivial inference involved in their determination, photo-$z$ point 
estimates are being superseded by photo-$z$ probability distribution functions 
(PDFs).  However, analytic methods for utilizing these new data products in 
cosmological inference are still evolving.

This paper presents a novel approach to estimating the posterior distribution 
over $n(z)$ from a survey of galaxy photo-$z$ PDFs based upon a probabilistic 
graphical model of hierarchical inference.  We present the Cosmological 
Hierarchical Inference with Probabilistic Photometric Redshifts (\chippr) code 
implementing this technique, as well as its validation on mock data and testing 
on the \texttt{Buzzard} simulations.  \chippr yields a more accurate 
characterization of $n(z)$ containing information beyond the best-fit estimator 
produced by traditional procedures.  The publicly available code is easily 
extensible to other one-point statistics that depend on redshift.

\end{abstract}

\keywords{catalogs --- cosmology: cosmological parameters --- galaxies: 
statistics --- gravitational lensing: weak --- methods: analytical --- methods: 
data analysis --- methods: statistical --- techniques: photometric}

\section{Introduction}
\label{sec:introduction}

\chippr aims to answer the following questions:

\begin{itemize}
	\item How are photo-$z$ PDFs currently used in cosmology?
	\item Why should we challenge existing methods?
	\item How should photo-$z$ PDFs be used in inference?
	\item How does the result of \chippr compare to established estimators 
of $n(z)$?
\end{itemize}

\section{Method}
\label{sec:method}

\subsection{Model}
\label{sec:model}

This is where I will present the DAG and corresponding math, along with an 
explicit discussion of the assumptions that must be satisfied in order for the 
\chippr approach to be valid.

\subsection{Implementation}
\label{sec:implementation}

I will describe the \chippr code here.

\subsection{Metrics}
\label{sec:metrics}

I will discuss the metrics that I use in \ref{sec:validation} and include in 
the release version.

\section{Validation}
\label{sec:validation}

This is where the setup and results of the simulated test cases will be 
presented.

\subsection{Variations on Photo-$z$ Likelihoods}
\label{sec:likelihoods}

I think this is too many test cases!

\begin{itemize}
	\item A fiducial case of Gaussian likelihoods with narrow, 
redshift-independent intrinsic scatter and no other systematics; also the 
fiducial case with larger intrinsic scatter
	\item The fiducial case with catastrophic outliers as seen in 
template-based photo-$z$ estimators (attractor in the space of $z_{phot}$: some 
galaxies at a range of $z_{spec}$ map to the same $z_{phot}$ because their 
shared SED type does not have sufficiently strong features, leading galaxies of 
that type at many $z_{spec}$ to have the same colors)
	\item The fiducial case with catastrophic outliers as seen in 
training-based photo-$z$ estimators (attractor in the space of $z_{spec}$: some 
galaxies at a particular $z_{spec}$ map to a range of $z_{phot}$ because their 
shared SED features fall between filters, leading to many galaxies at 
particular $z_{spec}$ to have the same colors)
\end{itemize}

\subsection{Variations on Interim Priors}
\label{sec:priors}

I will show that a really wacky interim prior, something very far from the true 
$n(z)$, will imprint itself on the stacked estimator but will have no effect on 
\chippr's result.

\subsection{Variations on True Hyperparameters}
\label{sec:truth}

I will show that if $n(z)$ has significant features, stacking will not recover 
them but \chippr will.

\section{Application}
\label{sec:application}

I will apply \chippr to a realistic simulated dataset (probably 
\texttt{Buzzard}) so the results can be compared to the truth.  The photo-$z$ 
PDFs should be derived using the \texttt{SDSS}-approved $k$ nearest neighbors 
algorithm because it has the most straightforward interim prior.

\section{Conclusion}
\label{sec:conclusion}

I will review the questions from the introduction:

\begin{itemize}
	\item Existing methods have shortcomings that propagate to inaccuracies 
in characterizing the cosmological parameters.
	\item Photo-$z$ PDFs are probabilistic data products so must be handled 
in a mathematically consistent manner; \chippr is one such method, conditioned 
on some assumptions.
	\item In addition to coming with its own error distribution, the $n(z)$ 
estimator produced by \chippr is more accurate than established estimators; 
furthermore, propagation of the \chippr result leads to a quantifiable 
improvement in the constraints on cosmological parameters.
\end{itemize}

I may briefly discuss some obvious extensions of \chippr, such as a fully 
hierarchical inference of the cosmological parameters that avoids tomographic 
binning based on photo-$z$ point estimates.



\begin{acknowledgements}
AIM thanks Phil Marshall for significant coaching on code development and paper 
preparation, Mohammadjavad Vakili for insightful input on statistics, Geoffrey 
Ryan for assistance in debugging, and Boris Leistedt for helpful comments 
provided in the preparation of this paper.
\end{acknowledgements}

\bibliographystyle{apj}
\bibliography{references}

\end{document}